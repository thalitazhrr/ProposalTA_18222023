% ==========================================
% BAB I PENDAHULUAN
% ==========================================
\chapter{PENDAHULUAN}
\label{chap:pendahuluan}
% --- Latar Belakang ---
\section{Latar Belakang}

Transformasi digital di sektor bisnis telah mengubah cara perusahaan menganalisis dan memanfaatkan data untuk pengambilan keputusan strategis. \textit{Business Intelligence} (BI) kini menjadi tulang punggung operasional organisasi modern yang ingin tetap kompetitif di pasar global. Pasar global \textit{Business Intelligence} diproyeksikan mencapai USD 36{,}82 miliar pada tahun 2025 dan tumbuh menjadi USD 116{,}25 miliar pada tahun 2033, dengan tingkat pertumbuhan tahunan gabungan (CAGR) sebesar 14{,}98\% \autocite{Straits2024}. Pertumbuhan ini menunjukkan bahwa organisasi di seluruh dunia semakin menyadari nilai strategis sistem analitik berbasis data untuk meningkatkan efisiensi operasional dan daya saing.

Di Indonesia, adopsi teknologi kecerdasan buatan (\textit{Artificial Intelligence}/AI) dan digitalisasi juga mengalami akselerasi signifikan. Pemerintah Indonesia memproyeksikan bahwa AI akan menghasilkan manfaat ekonomi hingga USD 366 miliar selama dekade mendatang \autocite{Intimedia2024}. Selain itu, menurut laporan \textit{East Ventures—Digital Competitiveness Index} 2025, lebih dari 80\% pelaku bisnis di Indonesia telah mengintegrasikan AI ke dalam operasi bisnis, meskipun hanya sekitar 13\% yang menggunakan AI pada tingkat lanjutan \autocite{EastVC2025}.

Salah satu tantangan utama dalam implementasi BI adalah kompleksitas akses dan interpretasi data bagi pengguna nonteknis. Sistem BI konvensional sering memerlukan keahlian khusus dalam bahasa kueri (seperti SQL) dan pemahaman mendalam tentang struktur basis data sehingga menjadi hambatan bagi manajer dan staf operasional yang membutuhkan \textit{insight} cepat untuk keputusan harian. Dalam mengatasi hal tersebut, teknologi \textit{Conversational AI} berbasis \textit{chatbot} muncul sebagai solusi yang menjanjikan. Pasar \textit{Conversational AI} global diperkirakan mencapai USD 49{,}7 miliar pada tahun 2025 dan tumbuh dengan CAGR sekitar 23{,}6\% hingga 2032 \autocite{Qaltivate2025}.

Implementasi \textit{chatbot} dalam konteks layanan pelanggan dan operasional internal telah menunjukkan hasil yang positif. Ringkasan industri menunjukkan bahwa \textit{chatbot} dapat menjawab hingga 80\% pertanyaan standar pelanggan, mempercepat waktu respons, dan memangkas beban kerja agen manusia \autocite{Fullview2025}. Dalam konteks BI, \textit{chatbot} berfungsi sebagai antarmuka percakapan yang memungkinkan pengguna internal, seperti manajer penjualan, tim layanan pelanggan, dan analis bisnis, mengakses data analitik, laporan, dan prediksi melalui kueri bahasa alami tanpa keahlian teknis. Studi \autocite{Iqbal2021} menunjukkan bahwa penerapan \textit{natural language processing} (NLP) pada \textit{chatbot} meningkatkan efisiensi dan akurasi pelayanan pelanggan secara signifikan.

Teknologi NLP menjadi fondasi yang memungkinkan sistem memahami dan merespons kueri pengguna dalam bahasa alami. Pasar NLP global diproyeksikan tumbuh dari USD 35{,}43 miliar pada tahun 2024 menjadi USD 438{,}08 miliar pada tahun 2034, dengan CAGR sekitar 28{,}6\% \autocite{Precedence2025}. Dalam aplikasi BI, NLP memungkinkan sistem menginterpretasikan \textit{intent} pengguna, mengekstraksi entitas penting dari kueri, dan menghasilkan respons relevan berdasarkan konteks percakapan, baik melalui pendekatan \textit{rule-based} untuk kueri terstruktur maupun melalui model pembelajaran mesin untuk kueri yang lebih kompleks.

Selain kemampuan analisis deskriptif dan diagnostik, kebutuhan akan kemampuan prediktif dalam BI juga meningkat. Analisis prediktif memungkinkan organisasi mengantisipasi permintaan layanan pelanggan, merencanakan alokasi sumber daya, dan mengidentifikasi potensi \textit{churn} sebelum terjadi. Dalam konteks ini, permintaan layanan pelanggan didefinisikan sebagai minat, pesanan, dan kebutuhan pelanggan terhadap produk atau layanan yang ditawarkan perusahaan, yang dapat diukur melalui data pemesanan, target penjualan, tingkat \textit{churn}, dan pendapatan pelanggan. Pasar \textit{time series forecasting} secara global diperkirakan tumbuh dari sekitar USD 11{,}17 miliar (2024) menjadi USD 36{,}9 miliar pada 2032, dengan CAGR \mbox{$\sim$16{,}12\%} \autocite{WiseGuyReports2024}.

Metode \textit{forecasting} yang lazim meliputi model statistik dan model \textit{deep learning} untuk menangkap pola temporal. Pada kasus prediksi \textit{customer churn} di industri telekomunikasi, baik Random Forest maupun LSTM efektif dalam prediksi \textit{churn}, tetapi kegunaannya bergantung pada sifat data dan batasan bisnis. LSTM lebih baik bila interaksi pelanggan historis dan tren kepuasan menjadi kunci dalam pengambilan keputusan \autocite{Mena2023}.

Integrasi tiga komponen utama, \textit{Rule-Based Query}, \textit{Natural Language Generation}, dan \textit{Time Series Forecasting}, ke dalam satu sistem BI berbasis \textit{chatbot} menawarkan solusi komprehensif bagi kebutuhan analisis dan prediksi di lingkungan bisnis modern. Sistem demikian berpotensi memperkuat pengambilan keputusan proaktif berbasis data serta meningkatkan efisiensi operasional internal perusahaan.

% --- Rumusan Masalah ---
\section{Rumusan Masalah}

Sebagian besar sistem \textit{Business Intelligence} untuk analisis indikator permintaan terkait pelanggan di lingkungan internal perusahaan pada saat ini masih mengandalkan pelaporan manual, belum terintegrasi dengan antarmuka percakapan berbasis \textit{chatbot}, dan minim pemanfaatan kecerdasan buatan untuk analisis prediktif serta interaksi berbasis bahasa alami. Kondisi ini memunculkan sejumlah permasalahan sebagai berikut.

\begin{enumerate}
  \item \textbf{Keterbatasan akses analitik berbasis percakapan}
  
  Tidak tersedia sistem yang mampu memproses dan memahami permintaan data serta analisis bisnis secara otomatis dari pengguna internal melalui percakapan dalam bahasa alami. Hal ini menyebabkan pegawai mengalami kendala dalam memperoleh \textit{insight} terkait pesanan (\textit{order}), pencapaian target penjualan, \textit{churn} pelanggan, serta pendapatan secara cepat dan intuitif untuk mendukung pengambilan keputusan harian.

  \item \textbf{Ketiadaan fitur prediksi yang mudah diakses pengguna nonteknis}
  
  Belum terdapat fitur prediksi indikator utama bisnis terkait pelanggan, seperti peramalan jumlah pesanan, \textit{churn} pelanggan, dan target pendapatan berbasis pembelajaran mesin atau peramalan deret waktu yang dapat diakses oleh pengguna nonteknis. Keadaan ini menyebabkan perencanaan kapasitas, evaluasi target, dan penetapan strategi bisnis perusahaan menjadi kurang responsif terhadap tren serta dinamika bisnis pada masa yang akan datang.

  \item \textbf{Minimnya integrasi \textit{rule-based query} dan \textit{natural language generation}}
  
  Integrasi \textit{rule-based query} dan pemrosesan bahasa alami (\textit{natural language generation}) dalam sistem BI internal masih sangat terbatas sehingga proses pencarian data, penyaringan metrik utama, serta penyusunan laporan kinerja bisnis masih harus dilakukan secara manual dan berulang. Keadaan tersebut menimbulkan keterlambatan pengambilan keputusan serta penyajian data yang kurang selaras dengan kebutuhan analisis oleh manajemen.
\end{enumerate}

Bagaimana merancang dan mengembangkan sistem \textit{Business Intelligence} berbasis \textit{chatbot} yang terintegrasi dengan teknologi \textit{rule-based query}, pemrosesan bahasa alami dengan (\textit{natural language generation}), dan algoritma peramalan deret waktu sehingga seluruh pengguna internal perusahaan dapat menganalisis, memantau, dan memprediksi indikator bisnis pelanggan, seperti pesanan, target, \textit{churn}, dan pendapatan melalui antarmuka percakapan, serta memperoleh \textit{insight} berbasis data yang akurat, efisien, dan relevan untuk mendukung pengambilan keputusan bisnis dan perencanaan strategis perusahaan?

% --- Tujuan ---
\section{Tujuan}

Tujuan utama tugas akhir ini adalah merancang dan mengembangkan sistem \textit{Business Intelligence} berbasis \textit{chatbot} yang terintegrasi dengan teknologi \textit{rule-based query}, pemrosesan bahasa alami (\textit{Natural Language Generation}/NLG), dan algoritma peramalan deret waktu (\textit{time series forecasting}) untuk memfasilitasi pengguna internal perusahaan dalam menganalisis dan memprediksi indikator bisnis pelanggan.

Secara terperinci, tujuan yang hendak dicapai dalam tugas akhir ini adalah sebagai berikut.
\begin{enumerate}
  \item Merancang arsitektur sistem \textit{Business Intelligence} berbasis \textit{chatbot} yang mampu memproses permintaan analisis data dan memberikan \textit{insight} berbasis data kepada pengguna internal perusahaan secara otomatis.
  \item Mengimplementasikan modul pemrosesan bahasa alami sehingga pengguna internal dapat mengajukan pertanyaan dan permintaan analisis bisnis menggunakan bahasa alami melalui antarmuka \textit{chatbot}. 
  \item Mengimplementasikan modul mekanisme \textit{rule-based query} untuk menangani permintaan data yang bersifat terstruktur dan berulang dengan respons yang cepat dan konsisten.
  \item Mengintegrasikan model peramalan deret waktu (\textit{time series forecasting}) untuk menghasilkan prediksi terhadap indikator bisnis pelanggan, seperti volume pesanan, pencapaian target penjualan, tingkat \textit{churn} pelanggan, dan proyeksi pendapatan.
  \item Mengintegrasikan model yang menghasilkan bahasa tertulis atau lisan alami dari data terstruktur.
  \item Merancang antarmuka percakapan yang dapat menampilkan percakapan.
  \item Menguji dan mengevaluasi kinerja sistem dari segi akurasi prediksi, kecepatan respons, serta kemudahan penggunaan (\textit{usability}) oleh pengguna internal perusahaan.
\end{enumerate}

% --- Batasan Masalah ---
\section{Batasan Masalah}

Dalam memfokuskan ruang lingkup penelitian dan memastikan kelayakan pelaksanaan tugas akhir, batasan-batasan penelitian ditetapkan sebagai berikut.
\begin{enumerate}
  \item Sistem dirancang khusus untuk pengguna internal perusahaan (manajer, analis bisnis, dan staf operasional), bukan untuk pelanggan eksternal.

  \item Sistem \textit{chatbot} hanya mendukung interaksi dalam Bahasa Indonesia.

  \item Sistem menggunakan data yang  terbatas pada data historis pelanggan yang mencakup informasi pesanan (\textit{order}), target penjualan, tingkat \textit{churn}, dan pendapatan dalam kurun waktu minimal dua tahun terakhir.

  \item Sistem hanya menangani permintaan terkait analisis deskriptif, diagnostik, dan prediktif terhadap indikator bisnis pelanggan. Sistem tidak mencakup fungsi transaksional atau modifikasi data.

  \item Sistem menggunakan metode peramalan deret waktu berbasis deep learning yang dibatasi pada algoritma LSTM, GRU, Temporal Fusion Transformer (TFT), atau kombinasi dari algoritma-algoritma tersebut.

  \item Sistem dirancang untuk mengenali maksimal 20 hingga 30 \textit{intent} utama yang berkaitan dengan analisis dan prediksi indikator bisnis pelanggan.

  \item Sistem yang dikembangkan merupakan purwarupa mandiri (\textit{standalone}) dan belum terintegrasi secara penuh dengan sistem ERP (\textit{Enterprise Resource Planning}) atau CRM (\textit{Customer Relationship Management}) yang ada di perusahaan.

  \item Sistem diimplementasikan pada platform berbasis \textit{web} dengan menggunakan \textit{framework} yang umum untuk pengembangan \textit{chatbot}.

  \item Sistem dievaluasi melalui pengujian fungsional, pengukuran akurasi prediksi, analisis waktu respons, serta survei kepuasan pengguna dengan jumlah responden terbatas (minimal 20 hingga 30 pengguna internal).

  \item Sistem mengimplementasikan fitur keamanan dan privasi data dibatasi pada tingkat dasar, seperti autentikasi pengguna dan enkripsi koneksi, tanpa mencakup standar keamanan tingkat \textit{enterprise} seperti ISO 27001.
\end{enumerate}

% --- Metodologi ---
\section{Metodologi}

\begin{figure}[H] 
  \centering
  \includegraphics[width=0.46\textwidth,
                   height=0.32\textheight,
                   keepaspectratio]{image/CRISP-DM.jpg}
  \caption{Metodologi \textit{CRISP-DM}}
  \label{fig:crispdm}
\end{figure}

Metodologi pelaksanaan tugas akhir ini mengadopsi kerangka kerja CRISP-DM (\textit{Cross-Industry Standard Process for Data Mining}) untuk tahapan pengolahan data dan analitik. Pendekatan ini dipilih agar proses penelitian berjalan secara iteratif, fleksibel, dan tetap sistematis dalam menyelesaikan permasalahan yang telah dirumuskan.

\textbf{CRISP-DM} (\textit{Cross-Industry Standard Process for Data Mining}) adalah metodologi standar lintas industri untuk penambangan data yang menyediakan pendekatan terstruktur dalam enam fase utama, yaitu pemahaman bisnis (\textit{business understanding}), pemahaman data (\textit{data understanding}), persiapan data (\textit{data preparation}), pemodelan (\textit{modeling}), evaluasi (\textit{evaluation}), dan penyebaran (\textit{deployment}) (DataScience-PM, 2024). Metodologi ini bersifat iteratif, yang memungkinkan pengulangan fase jika diperlukan untuk meningkatkan kualitas hasil. Dalam penelitian ini, \textbf{CRISP-DM digunakan} untuk mengelola aspek penambangan data dan analitik prediktif (khususnya pengembangan model peramalan deret waktu dan NLP).

\subsection{Tahapan Investigasi dan Pengumpulan Fakta}

Tahapan ini bertujuan untuk mengumpulkan data dan informasi faktual sebagai landasan perumusan masalah. Kegiatan yang dilakukan meliputi:

\begin{enumerate}[label=\alph*.]
  \item \textbf{Observasi sistem yang ada}, yaitu pengamatan terhadap sistem \textit{Business Intelligence} yang saat ini digunakan organisasi untuk mengidentifikasi keterbatasan dan kebutuhan pengguna internal.
  
  \item \textbf{Wawancara dengan pemangku kepentingan}, dilakukan secara terstruktur dengan minimal lima hingga sepuluh pemangku kepentingan internal (manajer layanan pelanggan, analis data, staf operasional) untuk menggali kebutuhan fungsional dan ekspektasi terhadap sistem baru.
  
  \item \textbf{Survei kebutuhan pengguna}, disebarkan kepada minimal tiga puluh hingga lima puluh calon pengguna untuk mengidentifikasi kebutuhan informasi, preferensi interaksi, dan kebutuhan analitik.
  
  \item \textbf{Analisis data historis}, dilakukan untuk menelaah volume dan tren permintaan, pola musiman, serta kelengkapan dan kualitas data yang tersedia.
  
  \item \textbf{Dokumentasi temuan}, mengompilasi seluruh hasil observasi, wawancara, survei, dan analisis dalam bentuk laporan yang akan dilampirkan.
\end{enumerate}

\subsection{Studi Literatur}

Tahapan ini bertujuan untuk mengumpulkan, mengelompokkan, dan menyaring literatur yang relevan dengan topik penelitian. Proses yang dilakukan meliputi:

\begin{enumerate}[label=\alph*.]
  \item \textbf{Pencarian literatur sistematis} pada basis data akademik (IEEE Xplore, ScienceDirect, SpringerLink, ACM Digital Library, Google Scholar) menggunakan kata kunci terkait \textit{business intelligence}, \textit{chatbot}, \textit{natural language generation}, dan \textit{time series forecasting}. Kriteria inklusi adalah artikel telaah sejawat (\textit{peer-reviewed}) tahun 2020--2025 yang relevan dengan topik.
  
  \item \textbf{Pengelompokan literatur} berdasarkan tema utama, yaitu konsep \textit{Business Intelligence}, teknologi \textit{chatbot} dan \textit{conversational AI}, pemrosesan bahasa alami, metode peramalan deret waktu, integrasi sistem, dan studi kasus implementasi.
  
  \item \textbf{Penapisan dan analisis kualitas} literatur berdasarkan faktor dampak, jumlah sitasi, metodologi, dan relevansi terhadap penelitian.
  
  \item \textbf{Sintesis literatur} untuk membangun kerangka teoretis, mengidentifikasi praktik terbaik dan teknologi terkini, serta menyusun tinjauan pustaka pada Bab II.
  
  \item \textbf{Dokumentasi proses} dalam bentuk matriks literatur dan pengelolaan referensi menggunakan \textit{reference manager}.
\end{enumerate}

\subsection{Perancangan Sistem}

Tahapan ini meliputi perancangan arsitektur dan komponen sistem secara terperinci, mencakup:

\begin{enumerate}[label=\alph*.]
  \item \textbf{Perancangan arsitektur sistem}, yaitu penyusunan arsitektur berlapis yang mengintegrasikan \textit{chatbot}, pemrosesan bahasa alami, mesin \textit{rule-based query}, dan peramalan deret waktu, serta penentuan teknologi dan kerangka kerja yang akan digunakan.
  
  \item \textbf{Perancangan modul \textit{chatbot}}, meliputi komponen pemahaman bahasa alami (\textit{Natural Language Understanding}), manajemen dialog (\textit{Dialogue Management}), pembangkitan bahasa alami (\textit{Natural Language Generation}), serta pendefinisian \textit{intent} dan entitas yang akan dikenali.
  
  \item \textbf{Pengimplementasian \textit{rule-based query}}, yaitu pendefinisian aturan untuk kueri terstruktur dan berulang serta pemetaan antara \textit{intent} dengan templat kueri.
  
  \item \textbf{Perancangan model peramalan deret waktu}, meliputi penentuan variabel target, perancangan praolah data, pemilihan arsitektur model (ARIMA, SARIMA, LSTM), dan strategi evaluasi model.
  
  \item \textbf{Perancangan basis data dan gudang data}, yaitu perancangan skema basis data untuk menyimpan data pelanggan, hasil prediksi, dan data pelatihan NLP, serta perancangan proses ETL (\textit{Extract, Transform, Load}).
  
  \item \textbf{Perancangan antarmuka pengguna}, meliputi pembuatan \textit{wireframe} dan \textit{mockup} untuk antarmuka \textit{chatbot}.
\end{enumerate}

\subsection{Implementasi Sistem}

Tahapan ini meliputi pengembangan sistem berdasarkan rancangan yang telah ditetapkan, mencakup:

\begin{enumerate}[label=\alph*.]
  \item \textbf{Pengembangan \textit{backend}}, yaitu implementasi mesin \textit{rule-based query}, model NLP untuk klasifikasi \textit{intent} dan pengenalan entitas, model peramalan deret waktu, serta penyediaan API \textit{backend}.
  
  \item \textbf{Pengembangan \textit{frontend}}, meliputi implementasi antarmuka \textit{chatbot}.
  
  \item \textbf{Pengembangan basis data}, yaitu implementasi skema basis data, proses ETL untuk integrasi data, dan \textit{data pipeline} untuk pemrosesan data.
  
  \item \textbf{Integrasi komponen}, meliputi integrasi \textit{chatbot} dengan mesin \textit{rule-based query} dan NLP, integrasi model peramalan dengan sistem BI, serta implementasi mekanisme tembolok (\textit{caching}) untuk performa.
\end{enumerate}

\subsection{Pengujian dan Evaluasi}
Tahapan ini bertujuan menilai kinerja sistem secara menyeluruh—mencakup aspek fungsionalitas, akurasi, performa, dan kebergunaan.

\begin{enumerate}[label=\alph*.]
  \item \textbf{Pengujian unit}, yaitu setiap komponen diuji secara terpisah agar fungsinya terverifikasi dengan benar dan stabil.
  \item \textbf{Pengujian integrasi}, yaitu Interaksi antarmodul diverifikasi melalui skenario ujung-ke-ujung, dari masukan percakapan hingga keluaran berupa kueri yang dieksekusi, prediksi, narasi, dan visualisasi.
  \item \textbf{Pengujian fungsional}, yaitu Pemenuhan seluruh kebutuhan fungsional divalidasi menggunakan himpunan skenario representatif yang mencerminkan pertanyaan bisnis prioritas.
  \item \textbf{Evaluasi model NLP (klasifikasi intent multi\-kelas, satu label per ujaran)}, yaitu hasil dilaporkan sebagai \textit{top-1} dan \textit{top-\textit{k}} accuracy; \textit{precision}, \textit{recall}, dan \textit{F1-score} per kelas yang kemudian dirata-ratakan dengan pendekatan \textit{macro}, \textit{micro}, dan \textit{weighted}. Disajikan pula \textit{confusion matrix} berukuran $K\times K$, evaluasi ambang kepercayaan untuk mekanisme \textit{fallback}/\texttt{UNKNOWN}, serta ringkasan kalibrasi probabilitas.
  \item \textbf{Evaluasi model peramalan deret waktu}, yaitu protokol yang digunakan ialah \textit{rolling-origin backtesting} pada beberapa horizon prediksi dengan \textit{seasonal naive} sebagai garis dasar. Laporan mencakup MAE, RMSE, sMAPE, dan MASE, dilengkapi \textit{skill score} terhadap garis dasar. Hasil disajikan per horizon dan per entitas prioritas (median dan rentang antar kuartil).
  \item \textbf{Evaluasi performa sistem}, yaitu waktu respons, \textit{throughput}, dan latensi diukur pada beban realistis; \textit{load testing} dilakukan untuk menilai skalabilitas.
  \item \textbf{Evaluasi kebergunaan (usability)}, yaitu \textit{User Acceptance Testing} melibatkan sedikitnya 20–30 pengguna internal. Hasil dirangkum menggunakan skor \textit{System Usability Scale} (SUS) dan umpan balik kualitatif terkait kemudahan, kejelasan, dan alur kerja.
\end{enumerate}

\subsection{Dokumentasi dan Penyusunan Laporan}

Tahapan akhir meliputi dokumentasi dan penyusunan laporan tugas akhir, mencakup:

\begin{enumerate}[label=\alph*.]
  \item \textbf{Dokumentasi teknis}, yaitu penyusunan dokumentasi API, dokumentasi kode, panduan pengguna (\textit{user manual}), dan panduan penyebaran (\textit{deployment guide}).
  
  \item \textbf{Penyusunan laporan tugas akhir}, meliputi penulisan laporan sesuai format yang ditetapkan dengan penyuntingan dan pemeriksaan kelengkapan.
  
  \item \textbf{Persiapan presentasi}, meliputi pembuatan \textit{slide} presentasi, penyusunan skenario demonstrasi sistem, dan kurasi materi pendukung.
  
  \item \textbf{Revisi dan finalisasi}, yaitu revisi berdasarkan masukan pembimbing, finalisasi laporan dan sistem, serta persiapan sidang tugas akhir.
\end{enumerate}

\vspace{0.5cm}
\noindent Seluruh proses metodologi ini akan didokumentasikan dengan baik untuk memastikan keterbukaan, kemampuan direproduksi, dan akuntabilitas penelitian.