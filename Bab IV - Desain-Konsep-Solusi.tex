\chapter{DESAIN KONSEP SOLUSI}

Berdasarkan analisis komprehensif terhadap permasalahan yang dihadapi dan kebutuhan pengguna yang telah diuraikan pada bab-bab sebelumnya, penelitian ini mengajukan solusi terintegrasi berupa \textit{Business Intelligence Chatbot System} berbasis teknologi pembelajaran mendalam (\textit{deep learning}) yang menggabungkan tiga komponen utama, yaitu mekanisme pencocokan pola (\textit{pattern-based intent classification}) untuk pemahaman maksud pengguna, mesin kueri berbasis aturan (\textit{rule-based query engine}) untuk pengambilan \textit{data} terstruktur, serta model peramalan deret waktu (\textit{time series forecasting}) untuk analisis prediktif. Solusi yang diajukan dirancang dengan visi untuk menghilangkan hambatan aksesibilitas \textit{data} analitik bagi pengguna nonteknis melalui antarmuka percakapan yang intuitif dan responsif dalam bahasa alami, sekaligus meningkatkan kecepatan pengambilan keputusan melalui otomatisasi proses analisis dan pelaporan. Sistem ini mengintegrasikan kemampuan deskriptif, diagnostik, dan prediktif dalam satu \textit{platform} terpadu yang memanfaatkan infrastruktur \textit{Enterprise Data Warehouse} yang telah ada, sehingga meminimalkan \textit{disruption} terhadap operasional bisnis yang sedang berjalan dan mengurangi kebutuhan perubahan drastis pada lanskap teknologi eksisting.

Transformasi dari kondisi \textit{status quo} menuju solusi yang diusulkan melibatkan perubahan yang bersifat fundamental pada tiga aspek utama, yaitu arsitektur sistem yang bergeser dari pelaporan manual dan \textit{dashboard} statis menuju sistem otomatis berbasis percakapan dengan kemampuan pemrosesan mendekati \textit{real-time}, kemampuan analitik yang berkembang dari analisis retrospektif menjadi analisis prediktif yang bersifat proaktif, serta pengalaman pengguna yang beralih dari ketergantungan pada pengguna teknis yang memahami \textit{SQL} menjadi kemudahan akses bagi pengguna bisnis umum melalui interaksi berbasis bahasa alami. Dengan demikian, sistem diharapkan mampu menjembatani kesenjangan antara kebutuhan pengambilan keputusan yang cepat dan kompleksitas teknis yang selama ini menjadi penghalang utama dalam pemanfaatan penuh kapabilitas \textit{data} dan analitik di lingkungan organisasi.

Sistem analitik yang ada saat ini (\textit{as-is}) masih mengikuti pola operasional tradisional yang berfokus pada pelaporan manual dan eksplorasi \textit{data} yang terbatas. Proses dimulai dari identifikasi kebutuhan analitik oleh pengguna bisnis, dilanjutkan dengan penerjemahan kebutuhan tersebut menjadi kueri basis \textit{data} oleh analis atau tim teknis, eksekusi kueri terhadap \textit{Enterprise Data Warehouse}, pemrosesan dan transformasi \textit{data} untuk memenuhi standar kualitas serta format analisis, kemudian diakhiri dengan penyajian hasil dalam bentuk laporan atau visualisasi \textit{dashboard}. Keterbatasan utama dari pola \textit{as-is} ini mencakup waktu respons yang lama akibat proses multi-tahap dengan \textit{multiple touchpoint} antardepartemen sebelum \textit{insight} dapat dihasilkan, tingkat aksesibilitas yang rendah karena hanya pengguna dengan keahlian teknis yang mampu melakukan analisis secara mandiri, ketiadaan kemampuan prediktif karena sistem berfokus pada analisis historis tanpa dukungan \textit{forecasting} untuk pengambilan keputusan proaktif, serta efisiensi operasional yang rendah karena tim teknis harus mengalokasikan waktu signifikan untuk mengonversi permintaan analitik menjadi \textit{query} teknis alih-alih berfokus pada aktivitas bernilai tambah yang lebih strategis.

\section{Alur Proses Bisnis Sistem Analitik Saat Ini (\textit{As-Is})}

Proses bisnis sistem analitik saat ini direpresentasikan dalam notasi BPMN (\textit{Business Process Model and Notation}) yang menggambarkan interaksi antar aktor, keputusan-keputusan kritis, dan aliran data dalam organisasi. Proses dimulai dengan identifikasi kebutuhan analisis baru oleh pengguna bisnis, dilanjutkan dengan klarifikasi kebutuhan dan pengembangan \textit{query} oleh tim teknis, eksekusi \textit{query} pada \textit{data warehouse}, pemrosesan data hasil \textit{query}, pembangunan \textit{dashboard} atau laporan, \textit{review} dan validasi internal, iterasi perbaikan apabila diperlukan, serta diakhiri dengan publikasi hasil analisis kepada pengguna akhir. Di dalam alur tersebut terdapat beberapa \textit{decision point} (titik keputusan) yang bersifat kritis, antara lain: apakah permintaan dapat dipenuhi dengan \textit{dashboard existing} atau memerlukan \textit{custom query}, apakah data hasil \textit{query} sudah sesuai dengan ekspektasi pengguna atau memerlukan transformasi tambahan, serta apakah hasil analisis sudah valid untuk dipublikasikan atau masih memerlukan revisi. Proses \textit{as-is} ini bersifat iteratif dengan siklus \textit{feedback} yang panjang dan kurang efisien dalam memenuhi kebutuhan analitik yang dinamis dan beragam dari pengguna bisnis. Diagram BPMN yang menggambarkan proses \textit{as-is} disajikan pada Gambar~\ref{fig:bpmnasis}.

\begin{figure}[H] 
  \centering
  \includegraphics[width=1\textwidth,
                   height=1\textheight,
                   keepaspectratio]{image/bpmnasis}
  \caption{BPMN Sistem \textit{As-Is}}
  \label{fig:bpmnasis}
\end{figure}

Visualisasi proses \textit{as-is} menunjukkan kompleksitas dan iterasi yang terlibat dalam setiap siklus permintaan analitik. Pengguna bisnis harus menunggu \textit{multiple} tahap pemrosesan, \textit{review}, dan validasi sebelum hasil analisis dapat diakses. Aliran \textit{data} bersifat \textit{linear} dengan \textit{multiple handoff} antar departemen, sehingga menciptakan \textit{bottleneck} dan \textit{delay} yang signifikan. Pengalaman pengguna berpusat pada \textit{interface} \textit{dashboard} statis, tanpa kemampuan untuk melakukan eksplorasi \textit{data} secara \textit{interaktif} maupun berinteraksi melalui \textit{dialog} percakapan yang natural.

\section{Alur Proses Bisnis Sistem Analitik dengan Chatbot BI (\textit{To-Be})}

Proses bisnis sistem analitik yang diusulkan (\textit{to-be}) merepresentasikan tingkat \textit{automation} dan optimasi yang signifikan dibandingkan proses \textit{as-is}. Proses dimulai ketika pengguna mengajukan pertanyaan dalam bahasa alami kepada sistem \textit{chatbot}, yang kemudian dilanjutkan dengan proses \textit{login} ke sistem \textit{BI chatbot} untuk keperluan autentikasi dan otorisasi pengguna. Setelah autentikasi berhasil, sistem membangun halaman dialog \textit{chatbot} sebagai \textit{interface} utama interaksi antara pengguna dan sistem yang bersifat \textit{AI-powered}. Tahap pemrosesan dimulai dengan klasifikasi \textit{intent} berbasis pola untuk mengidentifikasi jenis pertanyaan yang diajukan, apakah bersifat deskriptif, diagnostik, atau prediktif. Berdasarkan jenis \textit{intent} tersebut, sistem melakukan \textit{routing} ke modul yang sesuai, yaitu untuk analisis deskriptif dan diagnostik, sistem menggunakan \textit{rule-based query engine} untuk menghasilkan dan mengeksekusi \textit{structured query} terhadap \textit{data warehouse}; sedangkan untuk analisis prediktif, sistem mengakses \textit{forecasting model} yang telah dilatih untuk menghasilkan prediksi sesuai dengan \textit{time horizon} dan \textit{parameter} yang relevan. Setelah hasil analisis diperoleh, sistem melakukan perhitungan \textit{KPI} dan agregasi \textit{data}, kemudian menerapkan \textit{natural language generation} berbasis \textit{template} (\textit{template-based NLG}) untuk mengonversi \textit{output} numerik menjadi narasi dalam bahasa alami yang koheren dan mudah dipahami. Respons yang telah diformat dalam Bahasa Indonesia kemudian ditampilkan kembali kepada pengguna melalui \textit{interface} \textit{chatbot}, dan pada tahap akhir pengguna dapat mengeksekusi \textit{dashboard} untuk melakukan \textit{ad-hoc analysis} atau \textit{KPI tracking} tambahan apabila diperlukan, atau proses diakhiri ketika analisis dinyatakan selesai.

Keunggulan proses \textit{to-be} dibandingkan proses \textit{as-is} terletak pada beberapa aspek utama \textbf{kecepatan eksekusi} yang jauh lebih tinggi (hitungan menit dibandingkan hari) berkat \textit{real-time query execution}, \textbf{aksesibilitas yang lebih luas} karena tidak lagi memerlukan keahlian teknis \textit{SQL} untuk mengakses \textit{insight}, \textbf{otomatisasi proses} yang mengurangi beban kerja manual tim teknis melalui pemanfaatan \textit{rule-based query} dan \textit{automated intent handling}, \textbf{kemampuan analitik yang diperluas} melalui integrasi komponen prediktif berbasis \textit{machine learning}, serta \textbf{\textit{user experience}} yang lebih baik melalui \textit{conversational interface} yang natural dan responsif. Secara keseluruhan, rancangan \textit{to-be} ini memungkinkan siklus permintaan hingga penyajian hasil analitik standar diselesaikan dalam waktu kurang dari satu menit untuk mayoritas kasus, sehingga lebih selaras dengan kebutuhan pengambilan keputusan yang cepat dan berbasis \textit{data driven}. Diagram \textit{BPMN} yang menggambarkan proses \textit{to-be} disajikan pada Gambar~\ref{fig:bpmntobe}.

\begin{figure}[H] 
  \centering
  \includegraphics[width=1\textwidth,
                   height=1\textheight,
                   keepaspectratio]{image/bpmntobe}
  \caption{BPMN Sistem \textit{To-Be}}
  \label{fig:bpmntobe}
\end{figure}

Visualisasi proses \textit{to-be} menunjukkan peningkatan yang signifikan dalam efisiensi dan responsivitas sistem. Pengguna dapat secara langsung mengajukan pertanyaan melalui \textit{interface} \textit{chatbot} tanpa perlu menunggu atau melibatkan tim teknis. Aliran \textit{data} bersifat paralel dengan otomatisasi penuh pada tahap pemrosesan, sehingga memungkinkan hasil analisis diterima dalam waktu yang sangat singkat. Pengalaman pengguna berpusat pada \textit{dialog} percakapan yang interaktif, dengan kemampuan untuk melakukan eksplorasi \textit{data} secara \textit{real-time} serta mengajukan \textit{follow-up queries} secara \textit{seamless}.

\section{Perbandingan Sistem Saat Ini (\textit{As-Is}) dan Sistem Baru (\textit{To-Be})}

Berdasarkan analisis proses bisnis yang telah diuraikan pada bagian sebelumnya, terdapat perbedaan fundamental antara sistem analitik saat ini (\textit{as-is}) dan sistem yang diusulkan (\textit{to-be}). Sistem \textit{as-is} mengandalkan pelaporan manual dengan keterlibatan tim teknis pada setiap tahap, sedangkan sistem \textit{to-be} mengotomatisasi sebagian besar proses melalui antarmuka percakapan berbasis \textit{chatbot}. Perbandingan kedua sistem berdasarkan aspek-aspek kritis disajikan pada Tabel~\ref{tab:perbandingan-asis-tobe}.

\begin{longtable}{|p{2.5cm}|p{5cm}|p{5cm}|}
\caption{Perbandingan Sistem \textit{As-Is} dan \textit{To-Be}} \label{tab:perbandingan-asis-tobe} \\
\hline
\textbf{Aspek} & \textbf{Sistem \textit{As-Is}} & \textbf{Sistem \textit{To-Be}} \\
\hline
\endfirsthead
\multicolumn{3}{c}{\tablename\ \thetable\ -- \textit{Lanjutan dari halaman sebelumnya}} \\
\hline
\textbf{Aspek} & \textbf{Sistem \textit{As-Is}} & \textbf{Sistem \textit{To-Be}} \\
\hline
\endhead
\hline \multicolumn{3}{r}{\textit{Bersambung ke halaman berikutnya}} \\
\endfoot
\hline
\endlastfoot

Waktu Respons & 2--5 hari untuk kueri kompleks & $<$2 menit untuk 95\% kueri \\
\hline

Antarmuka Pengguna & \textit{Dashboard} statis, laporan manual & Antarmuka percakapan (\textit{chatbot}) dalam Bahasa Indonesia \\
\hline

Ketergantungan Teknis & Memerlukan tim TI/analis data untuk setiap permintaan & Pengguna nonteknis dapat mengakses \textit{insight} secara mandiri \\
\hline

Aksesibilitas & Terbatas pada pengguna dengan keahlian SQL & Seluruh pengguna internal dapat mengakses melalui bahasa alami \\
\hline

Kemampuan Analitik & Deskriptif dan diagnostik (historis) & Deskriptif, diagnostik, dan prediktif (\textit{forecasting}) \\
\hline

Proses Pengambilan Data & Ekstraksi manual dari berbagai sistem, transformasi via \textit{spreadsheet} & Otomatis melalui \textit{rule-based query engine} \\
\hline

Kemampuan Prediktif & Tidak tersedia atau dilakukan manual secara periodik & Terintegrasi dengan model \textit{time series forecasting} (LSTM/GRU/TFT) \\
\hline

Format Respons & Tabel, grafik, laporan tertulis & Narasi bahasa alami (\textit{template-based NLG}) dengan visualisasi pendukung \\
\hline

Siklus \textit{Feedback} & Iteratif dan panjang dengan \textit{multiple handoff} & \textit{Real-time} dengan kemampuan \textit{follow-up queries} \\
\hline

Skalabilitas & Terbatas oleh kapasitas tim teknis & \textit{Horizontal scaling} melalui arsitektur \textit{microservices} \\
\hline

Ketersediaan & Jam kerja (bergantung pada tim teknis) & Diakses setiap saat (\textit{self-service}) \\
\hline

\end{longtable}


\section{Rancangan Solusi Keseluruhan}
\label{subsec:rancangan-solusi}

Rancangan solusi keseluruhan mengintegrasikan lima \textit{layer} arsitektur yang saling terhubung, yakni \textit{presentation layer}, \textit{application layer}, \textit{business logic layer}, \textit{data layer}, dan \textit{infrastructure layer}. Integrasi kelima \textit{layer} tersebut memastikan bahwa sistem memiliki tingkat fleksibilitas, skalabilitas, dan \textit{maintainability} yang tinggi untuk mendukung pertumbuhan bisnis serta evolusi kebutuhan analitik di masa depan.

\subsection{Komponen-Komponen Utama Arsitektur}

\textit{Presentation layer} mencakup antarmuka pengguna yang dibangun dengan teknologi \textit{web} modern, dengan fokus pada antarmuka \textit{chatbot} berbasis \textit{web} yang mendukung interaksi \textit{real-time} dengan \textit{backend} sistem. Antarmuka ini diimplementasikan menggunakan \textit{framework frontend} modern seperti \textit{React} dan \textit{Vue.js} dengan pemanfaatan \textit{WebSocket} untuk komunikasi \textit{duplex} yang memungkinkan percakapan dua arah tanpa \textit{delay} yang signifikan. \textit{Presentation layer} juga mencakup komponen untuk menampilkan visualisasi \textit{data}, \textit{dashboard} interaktif, dan \textit{report} yang terintegrasi dengan hasil analisis yang dihasilkan oleh \textit{backend}.

\textit{Application layer} mengenkapsulasi \textit{core logic} sistem yang terdiri atas tiga modul utama. Modul \textit{Natural Language Understanding} (NLU) bertanggung jawab untuk melakukan \textit{parsing} dan \textit{understanding} terhadap \textit{input} pengguna dalam bahasa alami. Modul \textit{Dialogue Management} mengelola \textit{state} percakapan dan menjaga \textit{coherence} dalam \textit{dialog} \textit{multi-turn}. Modul \textit{Response Generation} menghasilkan \textit{response} yang sesuai berdasarkan hasil analisis yang diterima dari \textit{business logic layer}. Di sisi lain, \textit{business logic layer} mencakup implementasi \textit{business rules} yang telah didefinisikan, \textit{intent classification engine}, \textit{rule-based query engine}, \textit{machine learning model} untuk \textit{forecasting}, serta \textit{template repository} untuk \textit{NLG}. Lapisan ini merupakan jantung sistem yang mengonversi \textit{intent} pengguna menjadi aksi teknis yang konkret dan menghasilkan \textit{insight} berbasis \textit{data}.

\textit{Data layer} mencakup koneksi dan interaksi dengan \textit{Enterprise Data Warehouse} yang menyimpan \textit{transactional data} dan \textit{master data}, serta \textit{cache layer} untuk menyimpan hasil \textit{query} yang sering diakses guna meningkatkan \textit{performance}. Lapisan ini juga mencakup \textit{repository} untuk menyimpan \textit{machine learning model} yang telah dilatih, \textit{metadata} mengenai \textit{data sources}, dan \textit{audit logs} untuk \textit{tracking} seluruh interaksi pengguna dengan sistem. \textit{Infrastructure layer} mencakup komponen teknis yang mendukung operasional sistem secara keseluruhan, antara lain \textit{web server} untuk \textit{hosting} aplikasi \textit{chatbot}, \textit{application server} untuk eksekusi \textit{application logic}, \textit{message broker} untuk \textit{async communication}, \textit{job scheduler} untuk \textit{batch processing}, \textit{monitoring} dan \textit{logging infrastructure}, serta \textit{security components} untuk \textit{authentication} dan \textit{authorization} pengguna.

\subsection{Arsitektur Sistem dan Aliran Data}

Arsitektur sistem secara keseluruhan dirancang dengan pendekatan \textit{microservices}, dengan setiap komponen fungsional, seperti \textit{chatbot service}, \textit{query service}, \textit{forecasting service}, dan NLG \textit{service}, diimplementasikan sebagai \textit{service} independen yang dapat diskalakan secara terpisah sesuai dengan kebutuhan beban kerja masing-masing. Komunikasi antarkomponen dilakukan melalui \textit{REST API} atau \textit{message queue} untuk memastikan \textit{loose coupling} dan tingkat \textit{high resilience} yang memadai terhadap gangguan.

Diagram arsitektur sistem yang lebih rinci disajikan pada Gambar~\ref{fig:arsitektur}, yang memperlihatkan integrasi kelima \textit{layer} dan aliran \textit{data} antarkomponen di dalam sistem. Arsitektur ini dirancang untuk mendukung:

\begin{enumerate}
    \item \textbf{Skalabilitas horizontal}, dengan setiap \textit{service} dapat diduplikasi dan di-\textit{load balance} sesuai dengan kebutuhan kapasitas.
    \item \textbf{\textit{Resilience}}, sehingga kegagalan pada satu \textit{service} tidak secara langsung menyebabkan kegagalan sistem secara keseluruhan.
    \item \textbf{\textit{Maintainability}}, karena setiap \textit{service} dapat dikembangkan, diuji, dan di-\textit{deploy} secara independen tanpa mengganggu komponen lain.
    \item \textbf{Interoperabilitas}, melalui kemampuan setiap \textit{service} untuk berinteraksi dengan sistem \textit{external} menggunakan \textit{standard API}.
    \item \textbf{\textit{Security}}, dengan penerapan \textit{multiple layer security} yang mencakup \textit{authentication}, \textit{authorization}, dan enkripsi \textit{data}.
\end{enumerate}

\begin{figure}[H] 
  \centering
  \includegraphics[width=1\textwidth,
                   height=1\textheight,
                   keepaspectratio]{image/arsitektur}
  \caption{Diagram Arsitektur Sistem Baru Secara Keseluruhan}
  \label{fig:arsitektur}
\end{figure}

Diagram arsitektur ini menggunakan sistem pewarnaan aliran (\textit{color--coded flow}) untuk membedakan jenis interaksi yang terjadi antar komponen, sehingga pembacaan terhadap alur kerja sistem menjadi lebih jelas dan tidak ambigu. Aliran berwarna biru menggambarkan proses internal dan komunikasi tingkat layanan, yaitu interaksi antarmodul di dalam \textit{backend layer} seperti \textit{authentication service}, \textit{NLU module}, \textit{dialogue manager}, \textit{rule-based query engine}, serta \textit{forecasting service}. Aliran ini memperlihatkan bagaimana setiap komponen bekerja secara terpadu dalam memproses permintaan pengguna. Aliran berwarna oranye merepresentasikan proses pengambilan data dan eksekusi \textit{analytical query}, khususnya komunikasi antara \textit{backend} dan \textit{Enterprise Data Warehouse (EDW)} melalui \textit{Oracle data source \& connection pool} untuk memperoleh data historis maupun metrik operasional yang relevan. Sementara itu, aliran berwarna ungu menunjukkan aliran data lintas lapisan dan propagasi hasil, yaitu bagaimana keluaran analitik, seperti hasil kueri, agregasi, dan prediksi—dikembalikan ke \textit{business logic layer} untuk diproses lebih lanjut melalui mekanisme templating dan \textit{response generation}. Aliran berwarna abu-abu menggambarkan operasi tingkat infrastruktur dan fungsi non-fungsional, seperti \textit{logging}, \textit{rate limiting}, \textit{request/response monitoring}, serta \textit{health checks}, yang dikelola oleh \textit{API infrastructure unit} guna menjaga keandalan, visibilitas, dan stabilitas sistem secara menyeluruh. Terakhir, aliran berwarna merah digunakan untuk menandakan jalur penanganan kesalahan (\textit{error handling}), di mana setiap kesalahan yang terjadi dikirimkan ke \textit{central error handler} dan kemudian diteruskan ke subsistem \textit{monitoring \& logging} untuk keperluan pelacakan dan diagnostik. Melalui pemisahan aliran ini, diagram mampu memberikan representasi yang lebih presisi mengenai bagaimana permintaan dari \textit{React.js frontend} diproses secara end-to-end, mulai dari autentikasi, analisis \textit{intent}, eksekusi aturan, perhitungan prediktif, pengambilan data ke EDW, hingga penyusunan dan pengiriman kembali respons kepada pengguna.

Dalam mencegah terjadinya latensi yang tinggi, aspek \textit{response time} memang ditetapkan sebagai salah satu \textit{metric of interest} dalam penelitian ini mengingat produk yang dikembangkan beroperasi pada level operasional dan harus mampu memberikan jawaban secara cepat terhadap permintaan analitik harian. Tantangan utama muncul dari karakteristik \textit{Enterprise Data Warehouse} (EDW) yang menyimpan data dalam volume besar, sehingga eksekusi \textit{query} kompleks berpotensi menimbulkan \textit{delay} signifikan apabila dieksekusi berulang-ulang. Untuk mengatasi hal tersebut, sistem menerapkan serangkaian mekanisme optimasi, termasuk \textit{rule-level caching} dengan pendekatan \textit{Time To Live} (TTL) yang memastikan hasil \textit{query} disimpan sementara dan digunakan kembali tanpa perlu mengakses EDW setiap kali pengguna mengajukan permintaan yang sama. Selain itu, setiap \textit{rule} dikategorikan berdasarkan tingkat kompleksitas dan prioritas eksekusinya, sehingga sistem dapat mengalokasikan sumber daya pemrosesan secara lebih efisien dan menghindari \textit{bottleneck} pada permintaan yang berat. Kombinasi dari strategi ini memungkinkan penurunan latensi secara konsisten dan memastikan bahwa sistem tetap responsif dalam konteks penggunaan operasional.