\chapter{RENCANA SELANJUTNYA}

\section{Rencana Implementasi dan Evaluasi}
\label{sec:rencana-implementasi}

Bab ini menyajikan rencana implementasi sistem \textit{Business Intelligence} berbasis \textit{rule-based chatbot} yang mengintegrasikan klasifikasi maksud berbasis pola, pembangkitan bahasa alami berbasis templat, dan peramalan deret waktu. Pembahasan mencakup daftar perangkat dan teknologi yang digunakan, estimasi kebutuhan sumber daya, linimasa pelaksanaan tugas akhir, serta desain pengujian dan evaluasi yang akan dilakukan terhadap setiap komponen sistem. Selain itu, bab ini juga membahas analisis risiko dan strategi mitigasi untuk mengantisipasi kendala yang mungkin terjadi selama proses pengembangan.

\subsection{Rencana Implementasi}
\label{subsec:rencana-implementasi}

Dalam perencanaan implementasi sistem \textit{Business Intelligence Chatbot} yang mengintegrasikan komponen \textit{rule-based query}, \textit{template-based Natural Language Generation}, dan \textit{time series forecasting}, telah disusun daftar perangkat lunak, \textit{framework}, dan pustaka yang diperlukan untuk mendukung proses pengembangan. Selain itu, dirancang pula linimasa pelaksanaan tugas akhir yang menggambarkan rencana pengerjaan tiap tahap berdasarkan metodologi CRISP-DM (\textit{Cross-Industry Standard Process for Data Mining}).

\subsubsection{Perangkat Lunak dan Teknologi}
\label{subsubsec:tools}

Dalam melakukan implementasi sistem, dibutuhkan beberapa perangkat lunak dan teknologi yang mendukung proses pengembangan. Perangkat pada Tabel~\ref{tab:tools} memiliki fungsi dan peran yang berbeda dalam mengimplementasikan sistem \textit{Business Intelligence Chatbot}.

\begin{longtable}{|c|p{2.5cm}|p{9cm}|}
\caption{Perangkat Lunak dan Teknologi Implementasi Sistem} \label{tab:tools} \\
\hline
\textbf{No} & \textbf{Perangkat} & \textbf{Deskripsi} \\
\hline
\endfirsthead
\multicolumn{3}{c}%
{\tablename\ \thetable\ -- \textit{Lanjutan dari halaman sebelumnya}} \\
\hline
\textbf{No} & \textbf{Perangkat} & \textbf{Deskripsi} \\
\hline
\endhead
\hline \multicolumn{3}{r}{\textit{Bersambung ke halaman berikutnya}} \\
\endfoot
\hline
\endlastfoot

1 & Python & Bahasa pemrograman utama yang digunakan untuk pengembangan \textit{backend} sistem, implementasi model peramalan deret waktu, serta modul pemrosesan bahasa alami. Python dipilih karena ekosistem pustaka yang kaya untuk \textit{data science} dan \textit{machine learning}. \\
\hline

2 & FastAPI & \textit{Framework} Python untuk membangun \textit{RESTful API} dengan performa tinggi dan dokumentasi otomatis. FastAPI digunakan untuk menyediakan layanan \textit{backend} yang menangani permintaan dari antarmuka \textit{chatbot} dan mengembalikan respons dalam format terstruktur. \\
\hline

3 & React.js & Pustaka JavaScript untuk membangun antarmuka pengguna yang interaktif dan responsif. React.js digunakan untuk mengembangkan \textit{frontend chatbot} berbasis web dengan fitur percakapan \textit{real-time}. \\
\hline

4 & PostgreSQL & Sistem manajemen basis data relasional yang digunakan untuk menyimpan data pengguna, konfigurasi sistem, log percakapan, dan metadata. PostgreSQL dipilih karena keandalan dan dukungan terhadap kueri kompleks. \\
\hline

5 & Oracle Database & \textit{Enterprise Data Warehouse} yang menjadi sumber data utama untuk analisis bisnis. Sistem terhubung ke Oracle Database melalui koneksi \textit{read-only} untuk mengeksekusi kueri analitik berbasis aturan. \\
\hline

6 & TensorFlow & \textit{Framework deep learning} yang digunakan untuk membangun dan melatih model peramalan deret waktu berbasis LSTM, GRU, dan \textit{Temporal Fusion Transformer} (TFT). Keras menyediakan antarmuka tingkat tinggi yang memudahkan eksperimen arsitektur model. \\
\hline

7 & PyTorch & \textit{Framework deep learning} alternatif yang digunakan untuk implementasi model TFT (\textit{Temporal Fusion Transformer}) dengan pustaka \texttt{pytorch-forecasting}. PyTorch menawarkan fleksibilitas dalam definisi arsitektur model kustom. \\
\hline

8 & Pandas & Pustaka Python untuk manipulasi dan analisis data tabular. Pandas digunakan pada tahap \textit{data preparation}, transformasi fitur, dan agregasi data sebelum proses pelatihan model maupun eksekusi kueri. \\
\hline

9 & NumPy & Pustaka Python untuk komputasi numerik yang menjadi fondasi operasi matematika pada pemrosesan data dan perhitungan metrik evaluasi model. \\
\hline

10 & Scikit-learn & Pustaka \textit{machine learning} Python yang digunakan untuk praolah data, ekstraksi fitur, perhitungan metrik evaluasi, dan implementasi model \textit{baseline} untuk perbandingan performa. \\
\hline

11 & NLTK / spaCy & Pustaka pemrosesan bahasa alami yang digunakan untuk tokenisasi, \textit{stemming}, \textit{lemmatization}, dan praolah teks input pengguna sebelum proses klasifikasi maksud. \\
\hline

12 & Regex & Modul ekspresi reguler Python yang digunakan untuk implementasi pencocokan pola (\textit{pattern matching}) dalam klasifikasi maksud berbasis aturan. \\
\hline

13 & Jinja2 & \textit{Template engine} Python yang digunakan untuk implementasi \textit{template-based Natural Language Generation}. Jinja2 memungkinkan definisi templat respons dengan \textit{placeholder} yang diisi secara dinamis berdasarkan data hasil kueri. \\
\hline

14 & Docker & Platform kontainerisasi yang digunakan untuk mengemas aplikasi dan dependensinya ke dalam kontainer yang portabel. Docker memastikan konsistensi lingkungan pengembangan dan \textit{deployment}. \\
\hline

15 & Git / GitHub & Sistem kontrol versi dan platform kolaborasi yang digunakan untuk manajemen kode sumber, pelacakan perubahan, dan dokumentasi proyek tugas akhir. \\
\hline

16 & Jupyter Notebook & Aplikasi \textit{open-source} yang digunakan untuk eksplorasi data, eksperimen model, dan dokumentasi proses analisis dalam format dokumen interaktif yang menggabungkan kode, visualisasi, dan narasi. \\
\hline

17 & Google Colaboratory & Platform \textit{cloud} dari Google yang menyediakan akses gratis ke GPU untuk pelatihan model \textit{deep learning}. Google Colab digunakan untuk melatih model peramalan deret waktu yang memerlukan komputasi intensif. \\
\hline

18 & Postman & Aplikasi untuk pengujian dan dokumentasi API. Postman digunakan untuk menguji \textit{endpoint} API \textit{backend} sistem dan memvalidasi respons sebelum integrasi dengan \textit{frontend}. \\
\hline

19 & Visual Studio Code & \textit{Integrated Development Environment} (IDE) yang digunakan untuk penulisan kode, \textit{debugging}, dan manajemen proyek pengembangan sistem. \\
\hline

\end{longtable}

\subsubsection{Kebutuhan Sumber Daya Komputasi}
\label{subsubsec:resources}

Dalam mendukung proses pelatihan model \textit{deep learning} dan pengembangan sistem, diperlukan sumber daya komputasi yang memadai. Pelatihan model peramalan deret waktu berbasis LSTM, GRU, dan TFT memerlukan akselerasi GPU untuk mempercepat proses komputasi. Google Colaboratory menyediakan akses gratis ke GPU T4 NVIDIA dengan batasan akses per sesi sebesar 15GB memori GPU dan durasi eksekusi maksimal 12 jam per sesi.

Dalam kebutuhan pelatihan model yang lebih intensif atau eksperimen dengan \textit{hyperparameter tuning} yang ekstensif, dapat dipertimbangkan pembelian unit komputasi tambahan pada Google Colab Pro dengan estimasi biaya sekitar \$10 untuk 100 \textit{computing units}. Alternatif lain adalah penggunaan layanan \textit{cloud computing} seperti Google Cloud Platform, Amazon Web Services, atau Microsoft Azure dengan konfigurasi instans GPU sesuai kebutuhan.

Dalam \textit{deployment} sistem produksi, diperlukan infrastruktur server dengan spesifikasi prosesor dengan minimal 4 \textit{cores}, memori RAM minimal 16GB, penyimpanan SSD minimal 100GB, dan koneksi jaringan yang stabil. Sistem dapat di-\textit{deploy} pada infrastruktur \textit{on-premise} perusahaan atau menggunakan layanan \textit{cloud} dengan pertimbangan keamanan dan privasi data internal.

\subsubsection{Linimasa Pelaksanaan Tugas Akhir}
\label{subsubsec:timeline}

Dalam mencapai target penyelesaian tugas akhir, maka diperlukan suatu linimasa untuk memastikan seluruh bagian dari penelitian dilakukan. Terdapat dua tahap pengerjaan tugas akhir, yaitu tahap 1 pada Tabel~\ref{tab:timeline1} serta linimasa tahap 2 pada Tabel~\ref{tab:timeline2}.

\begin{table}[H]
\centering
\caption{Linimasa Pelaksanaan Tugas Akhir Tahap 1 (September 2025 -- Januari 2026)}
\label{tab:timeline1}
\scriptsize
\begin{tabular}{|p{3cm}|c|c|c|c|c|c|c|c|c|c|c|c|c|c|c|c|c|c|c|c|}
\hline
\multirow{2}{*}{\textbf{Kegiatan}} & \multicolumn{4}{c|}{\textbf{Sep}} & \multicolumn{4}{c|}{\textbf{Okt}} & \multicolumn{4}{c|}{\textbf{Nov}} & \multicolumn{4}{c|}{\textbf{Des}} & \multicolumn{4}{c|}{\textbf{Jan}} \\
\cline{2-21}
& 1 & 2 & 3 & 4 & 1 & 2 & 3 & 4 & 1 & 2 & 3 & 4 & 1 & 2 & 3 & 4 & 1 & 2 & 3 & 4 \\
\hline
Pembuatan Laporan TA I & \cellcolor{blue!30} & \cellcolor{blue!30} & \cellcolor{blue!30} & \cellcolor{blue!30} & \cellcolor{blue!30} & \cellcolor{blue!30} & \cellcolor{blue!30} & \cellcolor{blue!30} & \cellcolor{blue!30} & \cellcolor{blue!30} & \cellcolor{blue!30} & \cellcolor{blue!30} & \cellcolor{blue!30} & \cellcolor{blue!30} & \cellcolor{blue!30} & \cellcolor{blue!30} & \cellcolor{blue!30} & & & \\
\hline
Investigasi dan Pengumpulan Fakta & \cellcolor{blue!30} & \cellcolor{blue!30} & & & & & & & & & & & & & & & & & & \\
\hline
\textit{Business Understanding} & \cellcolor{blue!30} & \cellcolor{blue!30} & \cellcolor{blue!30} & & & & & & & & & & & & & & & & & \\
\hline
Kajian Teori dan Studi Literatur & & \cellcolor{blue!30} & \cellcolor{blue!30} & \cellcolor{blue!30} & \cellcolor{blue!30} & \cellcolor{blue!30} & & & & & & & & & & & & & & \\
\hline
Pembuatan BAB I & & & & \cellcolor{blue!30} & \cellcolor{blue!30} & & & & & & & & & & & & & & & \\
\hline
Pembuatan BAB II & & & & & \cellcolor{blue!30} & \cellcolor{blue!30} & \cellcolor{blue!30} & \cellcolor{blue!30} & & & & & & & & & & & & \\
\hline
Pembuatan BAB III & & & & & & & & \cellcolor{blue!30} & \cellcolor{blue!30} & \cellcolor{blue!30} & \cellcolor{blue!30} & & & & & & & & & \\
\hline
Pembuatan BAB IV & & & & & & & & & & & \cellcolor{blue!30} & \cellcolor{blue!30} & \cellcolor{blue!30} & & & & & & & \\
\hline
Pembuatan BAB V & & & & & & & & & & & & & \cellcolor{blue!30} & \cellcolor{blue!30} & & & & & & \\
\hline
Finalisasi Laporan TA I & & & & & & & & & & & & & & \cellcolor{blue!30} & \cellcolor{blue!30} & \cellcolor{blue!30} & & & & \\
\hline
Seminar Laporan TA I & & & & & & & & & & & & & & & & & \cellcolor{blue!30} & \cellcolor{blue!30} & & \\
\hline
\end{tabular}
\end{table}

\begin{table}[H]
\centering
\caption{Linimasa Pelaksanaan Tugas Akhir Tahap 2 (Februari -- Juli 2026)}
\label{tab:timeline2}
\scriptsize
\begin{tabular}{|p{1.5cm}|c|c|c|c|c|c|c|c|c|c|c|c|c|c|c|c|c|c|c|c|c|c|c|c|}
\hline
\multirow{2}{*}{\textbf{Kegiatan}} & \multicolumn{4}{c|}{\textbf{Feb}} & \multicolumn{4}{c|}{\textbf{Mar}} & \multicolumn{4}{c|}{\textbf{Apr}} & \multicolumn{4}{c|}{\textbf{Mei}} & \multicolumn{4}{c|}{\textbf{Jun}} & \multicolumn{4}{c|}{\textbf{Jul}} \\
\cline{2-25}
& 1 & 2 & 3 & 4 & 1 & 2 & 3 & 4 & 1 & 2 & 3 & 4 & 1 & 2 & 3 & 4 & 1 & 2 & 3 & 4 & 1 & 2 & 3 & 4 \\
\hline
Pembuatan Laporan TA II & \cellcolor{blue!30} & \cellcolor{blue!30} & \cellcolor{blue!30} & \cellcolor{blue!30} & \cellcolor{blue!30} & \cellcolor{blue!30} & \cellcolor{blue!30} & \cellcolor{blue!30} & \cellcolor{blue!30} & \cellcolor{blue!30} & \cellcolor{blue!30} & \cellcolor{blue!30} & \cellcolor{blue!30} & \cellcolor{blue!30} & \cellcolor{blue!30} & \cellcolor{blue!30} & \cellcolor{blue!30} & \cellcolor{blue!30} & \cellcolor{blue!30} & \cellcolor{blue!30} & & & & \\
\hline
\textit{Data Understanding} & \cellcolor{blue!30} & \cellcolor{blue!30} & & & & & & & & & & & & & & & & & & & & & & \\
\hline
\textit{Data Gathering} & \cellcolor{blue!30} & \cellcolor{blue!30} & \cellcolor{blue!30} & & & & & & & & & & & & & & & & & & & & & \\
\hline
\textit{Exploratory Data Analysis} & & & \cellcolor{blue!30} & \cellcolor{blue!30} & \cellcolor{blue!30} & & & & & & & & & & & & & & & & & & & \\
\hline
\textit{Data Preparation} & & & & & \cellcolor{blue!30} & \cellcolor{blue!30} & \cellcolor{blue!30} & & & & & & & & & & & & & & & & & \\
\hline
Kajian dan Penentuan Model & & & & & & & \cellcolor{blue!30} & \cellcolor{blue!30} & & & & & & & & & & & & & & & & \\
\hline
\textit{Modeling} & & & & & & & & \cellcolor{blue!30} & \cellcolor{blue!30} & \cellcolor{blue!30} & \cellcolor{blue!30} & \cellcolor{blue!30} & & & & & & & & & & & & \\
\hline
Implementasi Sistem & & & & & & & & & & & \cellcolor{blue!30} & \cellcolor{blue!30} & \cellcolor{blue!30} & \cellcolor{blue!30} & \cellcolor{blue!30} & & & & & & & & & \\
\hline
\textit{Evaluation} & & & & & & & & & & & & & & & \cellcolor{blue!30} & \cellcolor{blue!30} & \cellcolor{blue!30} & \cellcolor{blue!30} & & & & & & \\
\hline
Pembuatan BAB IV dan V & & & & & & & & & & & & & & & & & \cellcolor{blue!30} & \cellcolor{blue!30} & \cellcolor{blue!30} & & & & & \\
\hline
Finalisasi Laporan TA II & & & & & & & & & & & & & & & & & & & \cellcolor{blue!30} & \cellcolor{blue!30} & \cellcolor{blue!30} & & & \\
\hline
Sidang Akhir & & & & & & & & & & & & & & & & & & & & & & \cellcolor{blue!30} & \cellcolor{blue!30} & \\
\hline
\end{tabular}
\end{table}


\subsection{Desain Pengujian dan Evaluasi}
\label{subsec:evaluasi}

Dalam memastikan bahwa sistem yang dikembangkan memenuhi kebutuhan fungsional dan nonfungsional yang telah ditetapkan, dirancang strategi pengujian dan evaluasi yang komprehensif. Evaluasi dilakukan terhadap setiap komponen utama sistem, yaitu modul klasifikasi maksud berbasis pola, modul pembangkitan bahasa alami berbasis templat, dan modul peramalan deret waktu. Selain itu, dilakukan pula evaluasi sistem secara keseluruhan melalui pengujian integrasi dan pengukuran kepuasan pengguna.

\subsubsection{Evaluasi Klasifikasi Maksud Berbasis Pola}
\label{subsubsec:eval-intent}

Modul klasifikasi maksud (\textit{intent classification}) dievaluasi menggunakan metrik standar klasifikasi multi-kelas dengan mempertimbangkan bahwa sistem dirancang untuk mengenali 20--30 \textit{intent} utama yang berkaitan dengan analisis dan prediksi indikator bisnis pelanggan. Evaluasi dilakukan dengan pendekatan \textit{one-vs-all} dan agregasi metrik menggunakan strategi \textit{macro-averaging}, \textit{micro-averaging}, dan \textit{weighted-averaging}.

\paragraph{Metrik Evaluasi Klasifikasi \textit{Intent}}

Metrik evaluasi yang digunakan untuk mengukur performa klasifikasi maksud mencakup:

\begin{enumerate}
    \item \textbf{Akurasi (\textit{Accuracy})}: Proporsi prediksi yang benar dari keseluruhan prediksi. Target minimal adalah 85\%.
    \item \textbf{Presisi (\textit{Precision})}: Proporsi prediksi positif yang benar dari seluruh prediksi positif. Dihitung per kelas kemudian diagregasi.
    \item \textbf{Daya Ingat (\textit{Recall})}: Proporsi sampel positif yang berhasil diprediksi benar dari seluruh sampel positif aktual.
    \item \textbf{Skor-F1 (\textit{F1-Score})}: Rata-rata harmonik dari presisi dan daya ingat, memberikan keseimbangan antara kedua metrik.
\end{enumerate}

Formula perhitungan metrik agregasi adalah sebagai berikut.

\begin{equation}
\text{Presisi}_{\text{makro}} = \frac{1}{K} \sum_{i=1}^{K} \text{Presisi}_i
\end{equation}

\begin{equation}
\text{Daya Ingat}_{\text{makro}} = \frac{1}{K} \sum_{i=1}^{K} \text{Daya Ingat}_i
\end{equation}

\begin{equation}
\text{Skor-F1}_{\text{makro}} = \frac{1}{K} \sum_{i=1}^{K} \text{Skor-F1}_i
\end{equation}

dengan $K$ adalah jumlah kelas \textit{intent} dalam sistem.

\paragraph{Analisis Nilai Keyakinan (\textit{Confidence Score})}

Sistem menggunakan mekanisme pemberian nilai keyakinan dalam rentang 0 hingga 1 untuk setiap prediksi klasifikasi maksud. Ambang batas (\textit{threshold}) ditetapkan pada nilai 0,70 dengan ketentuan sebagai berikut.

\begin{enumerate}
    \item Jika nilai keyakinan $\geq$ 0,70: Prediksi dianggap valid dan sistem memberikan respons berdasarkan \textit{intent} yang teridentifikasi.
    \item Jika nilai keyakinan $<$ 0,70: Prediksi dianggap tidak cukup yakin dan sistem meminta klarifikasi kepada pengguna atau menawarkan alternatif \textit{intent} yang mungkin (\textit{fallback mechanism}).
\end{enumerate}

\paragraph{Target Performa Klasifikasi \textit{Intent}}


Berdasarkan studi literatur dan praktik terbaik industri untuk sistem klasifikasi \textit{intent} dalam \textit{chatbot} bisnis, target performa yang ditetapkan adalah:

\begin{enumerate}
    \item Skor-F1 Makro $\geq$ 0,85: Menjamin performa yang baik dan seimbang di seluruh kelas, termasuk kelas-kelas dengan sampel pelatihan yang lebih sedikit.
    \item Skor-F1 Tertimbang $\geq$ 0,90: Menjamin performa yang baik pada kelas-kelas mayoritas yang mewakili mayoritas pertanyaan pengguna dalam praktik.
    \item Akurasi Keseluruhan $\geq$ 0,88: Target akurasi minimal untuk sistem dapat diterima dalam produksi.
    \item \textit{Coverage Rate} $\geq$ 80\%: Persentase kueri yang mencapai \textit{threshold} keyakinan 0,70.
    \item Waktu Respons $<$ 100 milidetik: Dalam 95\% kueri agar memberikan pengalaman pengguna yang responsif.
\end{enumerate}

\subsubsection{Evaluasi Pembangkitan Bahasa Alami Berbasis Templat}
\label{subsubsec:eval-nlg}

Modul \textit{Natural Language Generation} (NLG) berbasis templat dievaluasi dari dua aspek utama: akurasi faktual (\textit{factual accuracy}) dan kualitas bahasa (\textit{linguistic quality}). Evaluasi dilakukan dengan kombinasi metrik otomatis dan penilaian manusia untuk mendapatkan gambaran komprehensif mengenai kualitas keluaran sistem.

\paragraph{Metrik Evaluasi NLG Otomatis}

Metrik otomatis yang digunakan untuk mengevaluasi kualitas respons NLG mencakup:

\begin{enumerate}
    \item \textbf{BLEU Score}: Mengukur kecocokan n-gram antara keluaran sistem dan teks rujukan. Formula dasar BLEU adalah:
    \begin{equation}
    \text{BLEU} = BP \cdot \exp\left(\sum_{n=1}^{N} w_n \log p_n\right)
    \end{equation}
    dengan $p_n$ adalah presisi n-gram, $w_n$ adalah bobot untuk setiap tingkat n-gram, dan $BP$ adalah faktor penalti panjang (\textit{brevity penalty}).
    
    \item \textbf{ROUGE Score}: Mengukur \textit{recall} kecocokan n-gram, khususnya ROUGE-L yang mengukur \textit{longest common subsequence} antara keluaran dan rujukan.
    
    \item \textbf{BERTScore}: Mengukur kesamaan semantik berbasis representasi vektor dari model BERT, lebih sensitif terhadap parafrase dan variasi leksikal.
    
    \item \textbf{Akurasi Faktual}: Persentase respons yang menyajikan data numerik dan informasi bisnis dengan benar sesuai hasil kueri. Target adalah 100\% akurasi faktual.
\end{enumerate}

\paragraph{Evaluasi Subjektif Berbasis Pengguna}

Selain metrik otomatis, dilakukan penilaian subjektif oleh evaluator manusia dengan skala 1--5 untuk dimensi berikut.

\begin{enumerate}
    \item \textbf{Kelancaran (\textit{Fluency})}: Seberapa lancar dan alami teks respons dibaca.
    \item \textbf{Kewajaran (\textit{Naturalness})}: Seberapa wajar respons jika dibandingkan dengan respons manusia.
    \item \textbf{Kejelasan (\textit{Clarity})}: Seberapa mudah respons dipahami oleh pengguna.
    \item \textbf{Kecukupan (\textit{Adequacy})}: Seberapa lengkap respons menjawab pertanyaan pengguna.
\end{enumerate}

\paragraph{Target Performa NLG}

Target performa untuk modul NLG berbasis templat adalah:

\begin{enumerate}
    \item Akurasi Faktual: 100\% (tidak ada kesalahan penyajian data numerik).
    \item BLEU Score $\geq$ 0,40 untuk respons yang memiliki templat rujukan.
    \item Skor rata-rata penilaian subjektif $\geq$ 4,0 dari skala 5,0 untuk setiap dimensi evaluasi.
    \item Variasi respons: Minimal 3 variasi templat untuk setiap jenis \textit{intent} untuk menghindari respons yang monoton.
\end{enumerate}

\subsubsection{Evaluasi Peramalan Deret Waktu}
\label{subsubsec:eval-forecasting}

Modul peramalan deret waktu dievaluasi menggunakan protokol \textit{rolling-origin backtesting} pada beberapa horizon prediksi dengan \textit{seasonal naive} sebagai garis dasar (\textit{baseline}). Model yang dievaluasi mencakup LSTM, GRU, dan \textit{Temporal Fusion Transformer} (TFT), serta kombinasi hibrida dari algoritma-algoritma tersebut.

\paragraph{Metrik Evaluasi \textit{Forecasting}}

Metrik evaluasi yang digunakan untuk mengukur performa model peramalan deret waktu mencakup:

\begin{enumerate}
    \item \textbf{MAE (\textit{Mean Absolute Error})}: Rata-rata nilai absolut selisih antara prediksi dan nilai aktual.
    \begin{equation}
    \text{MAE} = \frac{1}{n} \sum_{i=1}^{n} |y_i - \hat{y}_i|
    \end{equation}
    
    \item \textbf{RMSE (\textit{Root Mean Squared Error})}: Akar kuadrat dari rata-rata kuadrat selisih, lebih sensitif terhadap kesalahan besar.
    \begin{equation}
    \text{RMSE} = \sqrt{\frac{1}{n} \sum_{i=1}^{n} (y_i - \hat{y}_i)^2}
    \end{equation}
    
    \item \textbf{sMAPE (\textit{Symmetric Mean Absolute Percentage Error})}: Persentase kesalahan simetris yang mengatasi masalah pembagian nol.
    \begin{equation}
    \text{sMAPE} = \frac{100\%}{n} \sum_{i=1}^{n} \frac{|y_i - \hat{y}_i|}{(|y_i| + |\hat{y}_i|)/2}
    \end{equation}
    
    \item \textbf{MASE (\textit{Mean Absolute Scaled Error})}: Kesalahan yang diskalakan terhadap kesalahan \textit{naive forecast}, memungkinkan perbandingan antar-seri waktu.
    \begin{equation}
    \text{MASE} = \frac{\text{MAE}}{\frac{1}{n-1} \sum_{i=2}^{n} |y_i - y_{i-1}|}
    \end{equation}
\end{enumerate}

\paragraph{\textit{Skill Score} terhadap Garis Dasar}

Dalam mengukur peningkatan performa model terhadap garis dasar (\textit{seasonal naive}), dihitung \textit{skill score} dengan formula:

\begin{equation}
\text{\textit{Skill Score}} = 1 - \frac{\text{MAE}_{\text{model}}}{\text{MAE}_{\text{baseline}}}
\end{equation}

Nilai \textit{skill score} positif menunjukkan bahwa model mengungguli garis dasar, dengan nilai mendekati 1 menunjukkan peningkatan yang signifikan.

\paragraph{Validasi Silang Deret Waktu (\textit{Time Series Cross-Validation})}

Evaluasi model menggunakan strategi validasi silang khusus deret waktu (\textit{expanding window} atau \textit{sliding window}) untuk memastikan bahwa evaluasi tidak menggunakan data masa depan (\textit{data leakage}). Proses validasi mencakup:

\begin{enumerate}
    \item Pembagian data menjadi \textit{training set} (minimal 70\%), \textit{validation set} (15\%), dan \textit{test set} (15\%).
    \item Evaluasi pada beberapa horizon prediksi: 7 hari, 14 hari, 30 hari, dan 90 hari.
    \item Pelaporan hasil per horizon dan per entitas prioritas (median dan rentang antar-kuartil).
\end{enumerate}

\paragraph{Target Performa \textit{Forecasting}}

Target performa untuk model peramalan deret waktu adalah:

\begin{enumerate}
    \item MASE $<$ 1,0: Model harus mengungguli \textit{naive forecast}.
    \item sMAPE $<$ 15\% untuk prediksi jangka pendek (7--14 hari).
    \item sMAPE $<$ 25\% untuk prediksi jangka menengah (30--90 hari).
    \item \textit{Skill Score} $>$ 0,20 terhadap garis dasar \textit{seasonal naive}.
    \item Waktu inferensi $<$ 500 milidetik per permintaan prediksi.
\end{enumerate}

\subsubsection{Evaluasi Sistem Terintegrasi}
\label{subsubsec:eval-system}

Selain evaluasi per komponen, dilakukan pula evaluasi sistem secara keseluruhan melalui pengujian integrasi dan pengukuran performa \textit{end-to-end}.

\paragraph{Pengujian Integrasi}

Pengujian integrasi dilakukan untuk memverifikasi bahwa seluruh komponen sistem dapat bekerja sama dengan baik. Skenario pengujian mencakup:

\begin{enumerate}
    \item Pengujian alur lengkap dari input pengguna hingga respons akhir.
    \item Pengujian skenario percakapan multi-\textit{turn}.
    \item Pengujian mekanisme \textit{fallback} ketika klasifikasi maksud tidak yakin.
    \item Pengujian integrasi dengan \textit{Enterprise Data Warehouse}.
\end{enumerate}

\paragraph{Pengujian Performa Sistem}

Pengujian performa sistem mencakup:

\begin{enumerate}
    \item \textbf{Waktu Respons \textit{End-to-End}}: Target $<$ 2 detik untuk 95\% permintaan analitik standar dan $<$ 5 detik untuk permintaan peramalan.
    \item \textbf{\textit{Throughput}}: Kemampuan sistem menangani minimal 50 permintaan konkuren.
    \item \textbf{\textit{Load Testing}}: Pengujian beban untuk menilai skalabilitas sistem pada kondisi puncak.
\end{enumerate}

\paragraph{Evaluasi Kebergunaan (\textit{Usability})}

Evaluasi kebergunaan dilakukan melalui \textit{User Acceptance Testing} (UAT) yang melibatkan 20--30 pengguna internal perusahaan (manajer, analis bisnis, dan staf operasional). Evaluasi mencakup:

\begin{enumerate}
    \item \textbf{System Usability Scale (SUS)}: Kuesioner standar 10 pertanyaan untuk mengukur persepsi kebergunaan sistem. Target skor SUS $\geq$ 70 (kategori ``baik'').
    \item \textbf{Tingkat Penyelesaian Tugas (\textit{Task Completion Rate})}: Persentase pengguna yang berhasil menyelesaikan skenario tugas yang diberikan. Target $\geq$ 90\%.
    \item \textbf{Umpan Balik Kualitatif}: Pengumpulan masukan terkait kemudahan penggunaan, kejelasan respons, dan saran perbaikan.
\end{enumerate}

\subsection{Analisis Risiko dan Mitigasi}
\label{subsec:risiko}

Dalam memitigasi skenario dengan terdapat kondisi yang di luar ekspektasi awal, maka perlu dilakukan analisis risiko. Tabel~\ref{tab:risiko} menyajikan identifikasi risiko potensial beserta strategi mitigasi yang direncanakan.

\begin{longtable}{|c|p{3cm}|p{4cm}|p{4.5cm}|}
\caption{Analisis Risiko dan Strategi Mitigasi} \label{tab:risiko} \\
\hline
\textbf{No} & \textbf{Risiko} & \textbf{Dampak} & \textbf{Strategi Mitigasi} \\
\hline
\endfirsthead
\multicolumn{4}{c}%
{\tablename\ \thetable\ -- \textit{Lanjutan dari halaman sebelumnya}} \\
\hline
\textbf{No} & \textbf{Risiko} & \textbf{Dampak} & \textbf{Strategi Mitigasi} \\
\hline
\endhead
\hline \multicolumn{4}{r}{\textit{Bersambung ke halaman berikutnya}} \\
\endfoot
\hline
\endlastfoot

1 & Keterbatasan data historis untuk pelatihan model peramalan & Akurasi model peramalan tidak mencapai target yang ditetapkan & Menggunakan teknik augmentasi data, transfer learning dari model pre-trained, atau menyesuaikan horizon prediksi sesuai ketersediaan data \\
\hline

2 & Ketidakseimbang an distribusi kelas intent & Performa klasifikasi buruk pada kelas minoritas & Menerapkan teknik oversampling (SMOTE), undersampling, atau class weighting pada proses pelatihan \\
\hline

3 & Variasi bahasa alami pengguna di luar cakupan pola yang didefinisikan & Tingkat \textit{fallback} tinggi dan pengalaman pengguna menurun & Memperluas cakupan pola secara iteratif berdasarkan log percakapan dan umpan balik pengguna \\
\hline

4 & Keterbatasan sumber daya komputasi untuk pelatihan model & Waktu pelatihan model melebihi jadwal yang direncanakan & Menggunakan layanan cloud computing dengan GPU, mengoptimalkan arsitektur model, atau menggunakan teknik transfer learning \\
\hline

5 & Kendala akses ke \textit{Enterprise Data Warehouse} perusahaan & Data pengujian tidak representatif dengan kondisi produksi & Menggunakan data sintetis atau data anonim yang mewakili karakteristik data nyata untuk pengembangan dan pengujian \\
\hline

6 & Perubahan kebutuhan pengguna selama pengembangan & Fitur yang dikembangkan tidak sesuai ekspektasi pengguna akhir & Menerapkan pendekatan iteratif dengan evaluasi berkala bersama pemangku kepentingan dan penyesuaian prioritas \\
\hline

7 & Ketergantungan pada pustaka atau framework pihak ketiga & Ketidakcocokan versi atau \textit{deprecated features} menghambat pengembangan & Mendokumentasikan versi dependensi, menggunakan virtual environment, dan menyiapkan alternatif pustaka \\
\hline

8 & Responden UAT tidak mencukupi jumlah target & Evaluasi kebergunaan tidak valid secara statistik & Memperluas jangkauan rekrutmen responden, menawarkan insentif partisipasi, atau menyesuaikan target minimal responden \\
\hline

\end{longtable}